% Options for packages loaded elsewhere
\PassOptionsToPackage{unicode}{hyperref}
\PassOptionsToPackage{hyphens}{url}
\PassOptionsToPackage{dvipsnames,svgnames,x11names}{xcolor}
%
\documentclass[
  letterpaper,
  DIV=11,
  numbers=noendperiod]{scrartcl}

\usepackage{amsmath,amssymb}
\usepackage{iftex}
\ifPDFTeX
  \usepackage[T1]{fontenc}
  \usepackage[utf8]{inputenc}
  \usepackage{textcomp} % provide euro and other symbols
\else % if luatex or xetex
  \usepackage{unicode-math}
  \defaultfontfeatures{Scale=MatchLowercase}
  \defaultfontfeatures[\rmfamily]{Ligatures=TeX,Scale=1}
\fi
\usepackage{lmodern}
\ifPDFTeX\else  
    % xetex/luatex font selection
\fi
% Use upquote if available, for straight quotes in verbatim environments
\IfFileExists{upquote.sty}{\usepackage{upquote}}{}
\IfFileExists{microtype.sty}{% use microtype if available
  \usepackage[]{microtype}
  \UseMicrotypeSet[protrusion]{basicmath} % disable protrusion for tt fonts
}{}
\makeatletter
\@ifundefined{KOMAClassName}{% if non-KOMA class
  \IfFileExists{parskip.sty}{%
    \usepackage{parskip}
  }{% else
    \setlength{\parindent}{0pt}
    \setlength{\parskip}{6pt plus 2pt minus 1pt}}
}{% if KOMA class
  \KOMAoptions{parskip=half}}
\makeatother
\usepackage{xcolor}
\setlength{\emergencystretch}{3em} % prevent overfull lines
\setcounter{secnumdepth}{-\maxdimen} % remove section numbering
% Make \paragraph and \subparagraph free-standing
\ifx\paragraph\undefined\else
  \let\oldparagraph\paragraph
  \renewcommand{\paragraph}[1]{\oldparagraph{#1}\mbox{}}
\fi
\ifx\subparagraph\undefined\else
  \let\oldsubparagraph\subparagraph
  \renewcommand{\subparagraph}[1]{\oldsubparagraph{#1}\mbox{}}
\fi


\providecommand{\tightlist}{%
  \setlength{\itemsep}{0pt}\setlength{\parskip}{0pt}}\usepackage{longtable,booktabs,array}
\usepackage{calc} % for calculating minipage widths
% Correct order of tables after \paragraph or \subparagraph
\usepackage{etoolbox}
\makeatletter
\patchcmd\longtable{\par}{\if@noskipsec\mbox{}\fi\par}{}{}
\makeatother
% Allow footnotes in longtable head/foot
\IfFileExists{footnotehyper.sty}{\usepackage{footnotehyper}}{\usepackage{footnote}}
\makesavenoteenv{longtable}
\usepackage{graphicx}
\makeatletter
\def\maxwidth{\ifdim\Gin@nat@width>\linewidth\linewidth\else\Gin@nat@width\fi}
\def\maxheight{\ifdim\Gin@nat@height>\textheight\textheight\else\Gin@nat@height\fi}
\makeatother
% Scale images if necessary, so that they will not overflow the page
% margins by default, and it is still possible to overwrite the defaults
% using explicit options in \includegraphics[width, height, ...]{}
\setkeys{Gin}{width=\maxwidth,height=\maxheight,keepaspectratio}
% Set default figure placement to htbp
\makeatletter
\def\fps@figure{htbp}
\makeatother

\KOMAoption{captions}{tableheading}
\makeatletter
\@ifpackageloaded{caption}{}{\usepackage{caption}}
\AtBeginDocument{%
\ifdefined\contentsname
  \renewcommand*\contentsname{Table of contents}
\else
  \newcommand\contentsname{Table of contents}
\fi
\ifdefined\listfigurename
  \renewcommand*\listfigurename{List of Figures}
\else
  \newcommand\listfigurename{List of Figures}
\fi
\ifdefined\listtablename
  \renewcommand*\listtablename{List of Tables}
\else
  \newcommand\listtablename{List of Tables}
\fi
\ifdefined\figurename
  \renewcommand*\figurename{Figure}
\else
  \newcommand\figurename{Figure}
\fi
\ifdefined\tablename
  \renewcommand*\tablename{Table}
\else
  \newcommand\tablename{Table}
\fi
}
\@ifpackageloaded{float}{}{\usepackage{float}}
\floatstyle{ruled}
\@ifundefined{c@chapter}{\newfloat{codelisting}{h}{lop}}{\newfloat{codelisting}{h}{lop}[chapter]}
\floatname{codelisting}{Listing}
\newcommand*\listoflistings{\listof{codelisting}{List of Listings}}
\makeatother
\makeatletter
\makeatother
\makeatletter
\@ifpackageloaded{caption}{}{\usepackage{caption}}
\@ifpackageloaded{subcaption}{}{\usepackage{subcaption}}
\makeatother
\ifLuaTeX
  \usepackage{selnolig}  % disable illegal ligatures
\fi
\usepackage{bookmark}

\IfFileExists{xurl.sty}{\usepackage{xurl}}{} % add URL line breaks if available
\urlstyle{same} % disable monospaced font for URLs
\hypersetup{
  pdftitle={Exam 1 -Makeup},
  colorlinks=true,
  linkcolor={blue},
  filecolor={Maroon},
  citecolor={Blue},
  urlcolor={Blue},
  pdfcreator={LaTeX via pandoc}}

\title{Exam 1 -Makeup}
\author{}
\date{}

\begin{document}
\maketitle

\thispagestyle{empty}

\emph{Name:}

\emph{Notice: Calculators are not allowed. }

\begin{center}\rule{0.5\linewidth}{0.5pt}\end{center}

\subsection{Some formulas:}\label{some-formulas}

\begin{itemize}
\item
  The derivative of \(f(x)\) is defined by the formula: \[
  f'(x) = \lim_{h \to 0} \frac{f(x+h)-f(x)}{h}
  \]\\
\item
  An equation of the tangent line at \(x = a\) is \[
  y - f(a) = f'(a)(x-a)
  \]
\end{itemize}

\begin{center}\rule{0.5\linewidth}{0.5pt}\end{center}

\subsubsection{Problem 1. (5 points
each)}\label{problem-1.-5-points-each}

Find the following limits.

\[\lim_{x \to 2} \frac{x^2 - 4x + 4}{x^2 +5x + 1}\]

\hfill\break
\hfill\break
\hfill\break
\hfill\break

\thispagestyle{empty}

\[\lim_{x \to 2} \frac{x^2 - 3x + 2}{x^2 -6x + 8}\]

\hfill\break
\hfill\break
\hfill\break
\hfill\break
\hfill\break

\[\lim_{x \to \infty} \frac{2x^6 +4x^4 + 3x^3}{7x^5 + 2024}\]

\hfill\break
\hfill\break
\hfill\break
\hfill\break
\hfill\break

\[\lim_{x \to \infty} \frac{-2x^5 +4x + 3}{3x^{13} +5x + 6}\]

\hfill\break
\hfill\break
\hfill\break
\hfill\break
\hfill\break

\[\lim_{x \to \infty} \frac{x+1}{x^3 +5x + 6}\]

\hfill\break
\hfill\break
\hfill\break
\hfill\break
\hfill\break
\thispagestyle{empty}

\[\lim_{x \to 2} \frac{\sin 10x }{\sin 5x}\]

\hfill\break
\hfill\break
\hfill\break
\hfill\break
\hfill\break

\[\lim_{x \to 0} \frac{\sin 8x }{\sin 4x}\]

\hfill\break
\hfill\break
\hfill\break
\hfill\break
\hfill\break

\[\lim_{x \to 0} \frac{2x^3 + x ^2+ 3\sin x }{3x^3 + 3\sin 5x}\]

\hfill\break
\hfill\break
\hfill\break
\hfill\break
\hfill\break

\subsubsection{Problem 2 (5 points each)}\label{problem-2-5-points-each}

Find values of \(x\), if any, at which the function is not continuous.

\begin{enumerate}
\def\labelenumi{\alph{enumi}.}
\tightlist
\item
  \(f(x) = 3x^2 + \frac{1}{x-1} + 2024x +1\)
\end{enumerate}

\hfill\break
\hfill\break
\hfill\break
\hfill\break
\hfill\break

\thispagestyle{empty}

\begin{enumerate}
\def\labelenumi{\alph{enumi}.}
\setcounter{enumi}{1}
\tightlist
\item
  \(f(x) = x^2 + \frac{3}{(x - 2)(x-3)(x-4)} + 2024\)
\end{enumerate}

\hfill\break
\hfill\break
\hfill\break
\hfill\break
\hfill\break

\begin{enumerate}
\def\labelenumi{\alph{enumi}.}
\setcounter{enumi}{2}
\tightlist
\item
  \(f(x) = \frac{3}{x + 1} + \frac{x-1}{x^2 + 7x + 6}\)
\end{enumerate}

\hfill\break
\hfill\break
\hfill\break
\hfill\break
\hfill\break

\subsubsection{Problem 3. (5 points
each)}\label{problem-3.-5-points-each}

Find a value of the constant \(k\), if possible, that will make the
function continuous everywhere.

\begin{enumerate}
\def\labelenumi{\alph{enumi}.}
\tightlist
\item
  \begin{align*}
  f(x) =
   \left\{\begin{array}{lr}
     x^2, & x \le 2 \\
     x^2 - 3kx+1, & x > 2  
  \end{array}\right.
  \end{align*}
\end{enumerate}

\hfill\break
\hfill\break
\hfill\break
\hfill\break

\hfill\break
\hfill\break
\hfill\break
\hfill\break

\thispagestyle{empty}

\begin{enumerate}
\def\labelenumi{\alph{enumi}.}
\setcounter{enumi}{1}
\tightlist
\item
  \begin{align*}
  f(x) =
   \left\{\begin{array}{lr}
     3x^2 + 4x + 16, & x \le 0 \\
    - 9x + k^2, & x > 0
  \end{array}\right.
  \end{align*}
\end{enumerate}

\hfill\break
\hfill\break
\hfill\break
\hfill\break

\hfill\break
\hfill\break
\hfill\break
\hfill\break

\thispagestyle{empty}

\subsubsection{Problem 4. (17.5 points
each)}\label{problem-4.-17.5-points-each}

\begin{enumerate}
\def\labelenumi{\alph{enumi}.}
\tightlist
\item
  Use the definition of derivatives to find \(f'(x)\), and then find the
  tangent line to the graph of \(y = f(x)\) at \(x = 3\)
\end{enumerate}

\[
f(x) = -5x^2 + 6x + 2
\]

\hfill\break
\hfill\break
\hfill\break
\hfill\break

\hfill\break
\hfill\break
\hfill\break
\hfill\break

\hfill\break
\hfill\break
\hfill\break
\hfill\break

\hfill\break
\hfill\break
\hfill\break
\hfill\break

\thispagestyle{empty}

\begin{enumerate}
\def\labelenumi{\alph{enumi}.}
\setcounter{enumi}{1}
\tightlist
\item
  Use the definition of derivatives to find \(f'(x)\), and then find the
  tangent line to the graph of \(y = f(x)\) at \(x = 4\)
\end{enumerate}

\[
f(x) = \frac{4}{5x+1}
\]\\
\strut \\
\strut \\
\strut \\

\hfill\break
\hfill\break
\hfill\break
\hfill\break

\hfill\break
\hfill\break
\hfill\break
\hfill\break

\hfill\break
\hfill\break
\hfill\break
\hfill\break

\hfill\break
\hfill\break
\hfill\break
\hfill\break

\hfill\break
\hfill\break
\hfill\break
\hfill\break

\thispagestyle{empty}

c (Optional - 5 Points Extra Credits). Use the definition of derivatives
to find \(f'(x)\).

\[
f(x) = x^5
\]



\end{document}
