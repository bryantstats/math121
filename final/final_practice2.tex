% Options for packages loaded elsewhere
\PassOptionsToPackage{unicode}{hyperref}
\PassOptionsToPackage{hyphens}{url}
\PassOptionsToPackage{dvipsnames,svgnames,x11names}{xcolor}
%
\documentclass[
  letterpaper,
  DIV=11,
  numbers=noendperiod]{scrartcl}

\usepackage{amsmath,amssymb}
\usepackage{iftex}
\ifPDFTeX
  \usepackage[T1]{fontenc}
  \usepackage[utf8]{inputenc}
  \usepackage{textcomp} % provide euro and other symbols
\else % if luatex or xetex
  \usepackage{unicode-math}
  \defaultfontfeatures{Scale=MatchLowercase}
  \defaultfontfeatures[\rmfamily]{Ligatures=TeX,Scale=1}
\fi
\usepackage{lmodern}
\ifPDFTeX\else  
    % xetex/luatex font selection
\fi
% Use upquote if available, for straight quotes in verbatim environments
\IfFileExists{upquote.sty}{\usepackage{upquote}}{}
\IfFileExists{microtype.sty}{% use microtype if available
  \usepackage[]{microtype}
  \UseMicrotypeSet[protrusion]{basicmath} % disable protrusion for tt fonts
}{}
\makeatletter
\@ifundefined{KOMAClassName}{% if non-KOMA class
  \IfFileExists{parskip.sty}{%
    \usepackage{parskip}
  }{% else
    \setlength{\parindent}{0pt}
    \setlength{\parskip}{6pt plus 2pt minus 1pt}}
}{% if KOMA class
  \KOMAoptions{parskip=half}}
\makeatother
\usepackage{xcolor}
\setlength{\emergencystretch}{3em} % prevent overfull lines
\setcounter{secnumdepth}{-\maxdimen} % remove section numbering
% Make \paragraph and \subparagraph free-standing
\ifx\paragraph\undefined\else
  \let\oldparagraph\paragraph
  \renewcommand{\paragraph}[1]{\oldparagraph{#1}\mbox{}}
\fi
\ifx\subparagraph\undefined\else
  \let\oldsubparagraph\subparagraph
  \renewcommand{\subparagraph}[1]{\oldsubparagraph{#1}\mbox{}}
\fi


\providecommand{\tightlist}{%
  \setlength{\itemsep}{0pt}\setlength{\parskip}{0pt}}\usepackage{longtable,booktabs,array}
\usepackage{calc} % for calculating minipage widths
% Correct order of tables after \paragraph or \subparagraph
\usepackage{etoolbox}
\makeatletter
\patchcmd\longtable{\par}{\if@noskipsec\mbox{}\fi\par}{}{}
\makeatother
% Allow footnotes in longtable head/foot
\IfFileExists{footnotehyper.sty}{\usepackage{footnotehyper}}{\usepackage{footnote}}
\makesavenoteenv{longtable}
\usepackage{graphicx}
\makeatletter
\def\maxwidth{\ifdim\Gin@nat@width>\linewidth\linewidth\else\Gin@nat@width\fi}
\def\maxheight{\ifdim\Gin@nat@height>\textheight\textheight\else\Gin@nat@height\fi}
\makeatother
% Scale images if necessary, so that they will not overflow the page
% margins by default, and it is still possible to overwrite the defaults
% using explicit options in \includegraphics[width, height, ...]{}
\setkeys{Gin}{width=\maxwidth,height=\maxheight,keepaspectratio}
% Set default figure placement to htbp
\makeatletter
\def\fps@figure{htbp}
\makeatother

\KOMAoption{captions}{tableheading}
\makeatletter
\@ifpackageloaded{caption}{}{\usepackage{caption}}
\AtBeginDocument{%
\ifdefined\contentsname
  \renewcommand*\contentsname{Table of contents}
\else
  \newcommand\contentsname{Table of contents}
\fi
\ifdefined\listfigurename
  \renewcommand*\listfigurename{List of Figures}
\else
  \newcommand\listfigurename{List of Figures}
\fi
\ifdefined\listtablename
  \renewcommand*\listtablename{List of Tables}
\else
  \newcommand\listtablename{List of Tables}
\fi
\ifdefined\figurename
  \renewcommand*\figurename{Figure}
\else
  \newcommand\figurename{Figure}
\fi
\ifdefined\tablename
  \renewcommand*\tablename{Table}
\else
  \newcommand\tablename{Table}
\fi
}
\@ifpackageloaded{float}{}{\usepackage{float}}
\floatstyle{ruled}
\@ifundefined{c@chapter}{\newfloat{codelisting}{h}{lop}}{\newfloat{codelisting}{h}{lop}[chapter]}
\floatname{codelisting}{Listing}
\newcommand*\listoflistings{\listof{codelisting}{List of Listings}}
\makeatother
\makeatletter
\makeatother
\makeatletter
\@ifpackageloaded{caption}{}{\usepackage{caption}}
\@ifpackageloaded{subcaption}{}{\usepackage{subcaption}}
\makeatother
\ifLuaTeX
  \usepackage{selnolig}  % disable illegal ligatures
\fi
\usepackage{bookmark}

\IfFileExists{xurl.sty}{\usepackage{xurl}}{} % add URL line breaks if available
\urlstyle{same} % disable monospaced font for URLs
\hypersetup{
  pdftitle={Final Exam: Practice 2},
  colorlinks=true,
  linkcolor={blue},
  filecolor={Maroon},
  citecolor={Blue},
  urlcolor={Blue},
  pdfcreator={LaTeX via pandoc}}

\title{Final Exam: Practice 2}
\author{}
\date{}

\begin{document}
\maketitle

\thispagestyle{empty}

\emph{Name:}

\emph{Notice: Calculators are not allowed. }

\begin{center}\rule{0.5\linewidth}{0.5pt}\end{center}

\subsubsection{Problem}\label{problem}

Find the following limits.

\[\lim_{x \to 2} \frac{x^2 + 4x + 7}{x^2 +5x + 1}\]

\hfill\break
\hfill\break
\hfill\break
\hfill\break

\thispagestyle{empty}

\[\lim_{x \to 2} \frac{x^2 - 5x + 6}{x^2 -7x + 10}\]

\hfill\break
\hfill\break
\hfill\break
\hfill\break
\hfill\break

\[\lim_{x \to \infty} \frac{2x^5 +4x^4 + 3x^3}{7x^5 + 2024}\]

\hfill\break
\hfill\break
\hfill\break
\hfill\break
\hfill\break

\[\lim_{x \to \infty} \frac{-2x^8 +4x + 3}{3x^3 +5x + 6}\]

\hfill\break
\hfill\break
\hfill\break
\hfill\break
\hfill\break

\[\lim_{x \to \infty} \frac{7x + 3}{x^3 +5x + 6}\]

\hfill\break
\hfill\break
\hfill\break
\hfill\break
\hfill\break
\thispagestyle{empty}

\[\lim_{x \to 2} \frac{\sin x }{\sin 7x}\]

\hfill\break
\hfill\break
\hfill\break
\hfill\break
\hfill\break

\[\lim_{x \to 0} \frac{\sin 10x }{\sin 5x}\]

\hfill\break
\hfill\break
\hfill\break
\hfill\break
\hfill\break

\[\lim_{x \to 0} \frac{2x^2 + x+ \sin 3x }{3x^2 + 3\sin 5x}\]

\hfill\break
\hfill\break
\hfill\break
\hfill\break
\hfill\break

\subsubsection{Problem}\label{problem-1}

Find values of \(x\), if any, at which the function is not continuous.

\begin{enumerate}
\def\labelenumi{\alph{enumi}.}
\tightlist
\item
  \(f(x) = 3x^2 + \frac{x}{x-3} + 2024x +1\)
\end{enumerate}

\hfill\break
\hfill\break
\hfill\break
\hfill\break
\hfill\break

\thispagestyle{empty}

\begin{enumerate}
\def\labelenumi{\alph{enumi}.}
\setcounter{enumi}{1}
\tightlist
\item
  \(f(x) = x^2 + \frac{3}{(x - 1)(x-2)(x-3)} + 2024\)
\end{enumerate}

\hfill\break
\hfill\break
\hfill\break
\hfill\break
\hfill\break

\begin{enumerate}
\def\labelenumi{\alph{enumi}.}
\setcounter{enumi}{2}
\tightlist
\item
  \(f(x) = \frac{3}{x + 1} + \frac{x-1}{x^2 - 7x + 6}\)
\end{enumerate}

\hfill\break
\hfill\break
\hfill\break
\hfill\break
\hfill\break

\subsubsection{Problem}\label{problem-2}

Find a value of the constant \(k\), if possible, that will make the
function continuous everywhere.

\begin{enumerate}
\def\labelenumi{\alph{enumi}.}
\tightlist
\item
  \begin{align*}
  f(x) =
   \left\{\begin{array}{lr}
     x^2, & x \le 1 \\
     x^2 - 3kx+1, & x > 1  
  \end{array}\right.
  \end{align*}
\end{enumerate}

\hfill\break
\hfill\break
\hfill\break
\hfill\break

\hfill\break
\hfill\break
\hfill\break
\hfill\break

\thispagestyle{empty}

\begin{enumerate}
\def\labelenumi{\alph{enumi}.}
\setcounter{enumi}{1}
\tightlist
\item
  \begin{align*}
  f(x) =
   \left\{\begin{array}{lr}
     3x^2 + 4x + 1, & x \le 0 \\
    - 9x + k^2, & x > 0
  \end{array}\right.
  \end{align*}
\end{enumerate}

\hfill\break
\hfill\break
\hfill\break
\hfill\break

\hfill\break
\hfill\break
\hfill\break
\hfill\break

\thispagestyle{empty}

\subsubsection{Problem}\label{problem-3}

\begin{enumerate}
\def\labelenumi{\alph{enumi}.}
\tightlist
\item
  Use the definition of derivatives to find \(f'(x)\), and then find the
  tangent line to the graph of \(y = f(x)\) at \(x = 3\)
\end{enumerate}

\[
f(x) = 5x^2 - 6x + 1
\]

\hfill\break
\hfill\break
\hfill\break
\hfill\break

\hfill\break
\hfill\break
\hfill\break
\hfill\break

\hfill\break
\hfill\break
\hfill\break
\hfill\break

\hfill\break
\hfill\break
\hfill\break
\hfill\break

\thispagestyle{empty}

\begin{enumerate}
\def\labelenumi{\alph{enumi}.}
\setcounter{enumi}{1}
\tightlist
\item
  Use the definition of derivatives to find \(f'(x)\), and then find the
  tangent line to the graph of \(y = f(x)\) at \(x = 4\)
\end{enumerate}

\[
f(x) = \frac{5}{3x+1}
\]\\
\strut \\
\strut \\
\strut \\

\hfill\break
\hfill\break
\hfill\break
\hfill\break

\hfill\break
\hfill\break
\hfill\break
\hfill\break

\hfill\break
\hfill\break
\hfill\break
\hfill\break

\hfill\break
\hfill\break
\hfill\break
\hfill\break

\hfill\break
\hfill\break
\hfill\break
\hfill\break

\thispagestyle{empty}

c (Optional - 5 Points Extra Credits). Use the definition of derivatives
to find \(f'(x)\).

\[
f(x) = x^4
\]

\hfill\break
\hfill\break
\hfill\break
\hfill\break

\hfill\break
\hfill\break
\hfill\break
\hfill\break

\hfill\break
\hfill\break
\hfill\break
\hfill\break

\subsubsection{Problem}\label{problem-4}

\emph{(5 points each)} Find \(f'(x)\).

\[f(x) =  -\frac{2x}{3} + \frac{5x^2}{3} - \frac{1}{\sqrt[6]{x^3}} + \frac{1}{\sqrt{x}} + 2024x^2 + x + 2024\]

\hfill\break
\hfill\break
\hfill\break
\hfill\break

\[f(x) = (\sqrt{x}+1)(x^2+2x+1)\]

\hfill\break
\hfill\break
\hfill\break
\hfill\break
\hfill\break

\[f(x) = \frac{x^3+1}{x^3-1} \text{   (Simplify your answers.)}\]

\hfill\break
\thispagestyle{empty}

\[f(x) = 3x^2\cos x\]

\hfill\break
\hfill\break
\hfill\break
\hfill\break
\hfill\break

\[f(x) = \frac{\sin x}{x}\]

\hfill\break
\hfill\break
\hfill\break
\hfill\break
\hfill\break

\[f(x) = \sin \bigg(x + \sin x \bigg)\]

\hfill\break
\hfill\break
\hfill\break
\hfill\break
\hfill\break

\[f(x) = \cos^{2}x\]

\hfill\break
\hfill\break
\hfill\break
\hfill\break
\hfill\break

\thispagestyle{empty}

\[f(x) = \sin\bigg(x \cos x\bigg)\]

\[f(x) = \bigg(3\sin x - 2\cos x\bigg)^{2}\]

\hfill\break
\hfill\break
\hfill\break
\hfill\break
\hfill\break

\[f(x) = e^x +17^x - 2\log_{3}x + 8\ln x - \frac{3\log_2 x}{2} + \frac{\log_9 x}{3} + 2024x + 1\]

\hfill\break
\hfill\break
\hfill\break
\hfill\break
\hfill\break

\[f(x) = \log_{3}\bigg(\sqrt{x}+x^2+1\bigg)\]

\hfill\break
\hfill\break
\hfill\break
\hfill\break
\hfill\break

\thispagestyle{empty}

\[f(x) = 100^{\cos x - \sin x + 3x^2}\]

\hfill\break
\hfill\break
\hfill\break
\hfill\break

\[f(x) = e^{x\cos x}\]

\subsubsection{Problem}\label{problem-5}

\emph{(8 points each)}

\[
y + x^2y - x = 1
\]

\begin{enumerate}
\def\labelenumi{(\alph{enumi})}
\tightlist
\item
  Find \(dy/dx\) or \(y'\) by differentiating implicitly.
\end{enumerate}

\hfill\break
\hfill\break
\hfill\break
\hfill\break
\hfill\break
\hfill\break
\hfill\break
\hfill\break
\hfill\break
\hfill\break
\hfill\break
\hfill\break
\hfill\break

\begin{enumerate}
\def\labelenumi{(\alph{enumi})}
\setcounter{enumi}{1}
\tightlist
\item
  Solve the equation for y as a function of x, and find \(dy/dx\) from
  that equation.
\end{enumerate}

\thispagestyle{empty}

\hfill\break
\hfill\break
\hfill\break
\hfill\break
\hfill\break
\hfill\break
\hfill\break
\hfill\break
\hfill\break
\hfill\break
\hfill\break
\hfill\break
\hfill\break
\hfill\break
\hfill\break
\hfill\break

\begin{enumerate}
\def\labelenumi{(\alph{enumi})}
\setcounter{enumi}{2}
\tightlist
\item
  Find an equation for the tangent line at the point (1, 1)
\end{enumerate}

\hfill\break

\thispagestyle{empty}

\hfill\break
\hfill\break
\hfill\break
\hfill\break
\hfill\break
\hfill\break
\hfill\break
\hfill\break
\hfill\break
\hfill\break
\hfill\break
\hfill\break
\hfill\break

\subsubsection{Problem}\label{problem-6}

\emph{(5 points each)}

\begin{enumerate}
\def\labelenumi{(\alph{enumi})}
\tightlist
\item
  Find the local linear approximation of \(f(x) = \sqrt[4]{x}\) at
  \(x_0 = 1\)\\
\end{enumerate}

\thispagestyle{empty}

\hfill\break
\hfill\break
\hfill\break
\hfill\break
\hfill\break
\hfill\break
\hfill\break
\hfill\break
\hfill\break
\hfill\break
\hfill\break
\hfill\break
\hfill\break

\begin{enumerate}
\def\labelenumi{(\alph{enumi})}
\setcounter{enumi}{1}
\tightlist
\item
  Use the local linear approximation obtained in part (a) to approximate
  \(\sqrt[4]{1.1}\)
\end{enumerate}

\thispagestyle{empty}

\subsubsection{Problem}\label{problem-7}

Given that

\[f(x) =  x^3 - 6x^2 + 9x + 1\]

Find all the intervals where

\begin{enumerate}
\def\labelenumi{\alph{enumi}.}
\tightlist
\item
  \(f(x)\) is increasing
\item
  \(f(x)\) is decreasing
\item
  \(f(x)\) is concave upward
\item
  \(f(x)\) is concave downward
\end{enumerate}

\hfill\break
\hfill\break
\hfill\break
\hfill\break
\hfill\break
\hfill\break
\hfill\break
\hfill\break
\hfill\break
\hfill\break
\hfill\break
\hfill\break
\hfill\break
\hfill\break
\hfill\break
\hfill\break
\hfill\break
\hfill\break
\hfill\break
\hfill\break
\hfill\break
\hfill\break
\hfill\break
\hfill\break
\thispagestyle{empty}

\subsubsection{Problem}\label{problem-8}

Find all the relative extrema of

\[
f(x) = 2x^3 -9x^2 + 12x + 1
\]

\hfill\break
\hfill\break
\hfill\break
\hfill\break
\hfill\break
\hfill\break
\hfill\break
\hfill\break
\hfill\break
\hfill\break
\hfill\break
\hfill\break
\hfill\break
\hfill\break
\hfill\break
\hfill\break
\hfill\break
\hfill\break
\hfill\break
\hfill\break
\thispagestyle{empty}

\subsubsection{Problem}\label{problem-9}

Find an relative extrema of \(f(x) = x^4 - x^2 - x + 1\) using gradient
descent.

\hfill\break
\hfill\break
\hfill\break
\hfill\break
\hfill\break
\hfill\break
\hfill\break
\hfill\break
\hfill\break
\hfill\break
\hfill\break
\hfill\break
\hfill\break
\hfill\break
\hfill\break
\hfill\break
\hfill\break
\hfill\break
\hfill\break
\hfill\break
\hfill\break
\hfill\break
\hfill\break
\hfill\break
\hfill\break
\hfill\break
\hfill\break
\hfill\break
\hfill\break
\hfill\break
\thispagestyle{empty}

\subsubsection{Problem 4}\label{problem-4-1}

Find the absolute maximum and absolute minimum of
\(f(x) = 2x^3 - 15x^2 + 36x + 1\) on the interval {[}0, 1{]}.

\hfill\break
\hfill\break
\hfill\break
\hfill\break
\hfill\break
\hfill\break
\hfill\break
\hfill\break
\hfill\break
\hfill\break
\hfill\break
\hfill\break
\hfill\break
\hfill\break
\hfill\break
\hfill\break
\hfill\break
\hfill\break
\hfill\break
\hfill\break
\thispagestyle{empty}

\subsubsection{Problem}\label{problem-10}

The given equation has one (real) solution. Approximate the solution by
Newton's method.

\[x^3 + x - 1  = 0\]

\subsubsection{Problem}\label{problem-11}

Find the following

\[\int \bigg(x^7 - 2x^6 + 2x + 2024 \bigg) dx\]

\[\int \bigg( \sqrt x + x + \frac{1}{x}\bigg) dx\]

\[\int \bigg( 2^x + 2\sin x  - 3 \cos x + 1\bigg) dx\]

\[\int ( x  + 1)(x+2 ) dx\]

\subsubsection{Problem}\label{problem-12}

Calculate the area between \(f(x) = x^2 -4x + 3\) and x-axis bounded by
\(x = 0\) and \(x = 2\)



\end{document}
